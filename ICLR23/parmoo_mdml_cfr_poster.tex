\documentclass[a0paper,landscape]{baposter}

\usepackage{textcomp}
% \usepackage[T1]{fontenc}
\usepackage[utf8]{inputenc}
\usepackage{tgheros}
% \usepackage{tgbonum}
% \renewcommand*\familydefault{\sfdefault}
% \setmainfont{Arial}
\renewcommand{\familydefault}{\sfdefault}

\usepackage{calc}
\usepackage{amsmath}
\usepackage{amssymb}
\usepackage{mathtools}
\usepackage{relsize}
\usepackage{multirow}
\usepackage{bm}
\usepackage{enumitem}

%\usepackage{amsfonts}
%\usepackage{cmbright}
%\usepackage{comicsans}
\usepackage[cm]{sfmath}


\usepackage{graphicx}
\usepackage{multicol}

\pgfdeclarelayer{background}
\pgfdeclarelayer{foreground}
\pgfsetlayers{background,main,foreground}

\usepackage{tikz,pgfplots,pgfplotstable,gnuplot-lua-tikz}

\usetikzlibrary{shapes,arrows,decorations.markings,shadows,positioning}
% \usepackage{helvet}
%\usepackage{bookman}
%\usepackage{palatino}
\usepackage{enumitem}
\usepackage{tensor}
\newcommand\Perms[2]{\tensor[_{#1}]P{_{#2}}}
\newcommand{\captionfont}{\footnotesize}

\usepackage[vlined,titlenotnumbered]{algorithm2e}
\usepackage{minibox}
\usepackage{listings}
% Define custom style for AMPL code listings
\lstdefinelanguage{AMPL}{keywords={set,param,var,arc,integer,minimize,maximize,subject,to,node,
sum,in,Current,complements,integer,solve_result_num,IN,contains,less,suffix,INOUT,default,logical,
Infinity,dimen,max,symbolic,Initial,div,min,table,LOCAL,else,option,then,OUT,environ,setof,union,
all,exists,shell_exitcodeuntil,binary,forall,solve_exitcodewhile,by,if,solve_messagewithin,check,
solve_result,true,false,include},sensitive=true,comment=[l]{\#}}
\lstdefinestyle{AMPL}{
	language=AMPL,
	aboveskip=3mm,
	belowskip=3mm,
	showstringspaces=false,
	columns=flexible,
        %	keywordstyle=\bfseries,
        keywordstyle=\color{ArgonneLogoRed},
	breaklines=true,
	breakatwhitespace=true,
	tabsize=3,
}
% Define custom style for Python code listings
\definecolor{codegreen}{rgb}{0,0.6,0}
\definecolor{backcolour}{rgb}{0.95,0.95,0.92}
\lstdefinestyle{python}{
    language=python,
    numbers=none,
    backgroundcolor=\color{backcolour},
    commentstyle=\color{blue},
    keywordstyle=\color{magenta},
    stringstyle=\color{codegreen},
    basicstyle=\ttfamily\footnotesize,
    breakatwhitespace=false,
    breaklines=true,
    captionpos=b,
    keepspaces=true,
    showspaces=false,
    showstringspaces=false,
    showtabs=false,
    tabsize=3
}
\usepackage{tcolorbox}

\let\oldnl\nl% Store \nl in \oldnl
\newcommand{\nonl}{\renewcommand{\nl}{\let\nl\oldnl}}% Remove line number for one line

\renewcommand{\emph}{\textbf}

\graphicspath{{./images/}}
\setlist[itemize]{leftmargin=*}
\setlist[enumerate]{leftmargin=*}

%%%%%%%%%%%%%%%%%%%%%%%%%%%%%%%%%%%%%%%%%%%%%%%%%%%%%%%%%%%%%%%%%%%%%%%%%%%%%%%%
%%%% Some math symbols used in the text
%%%%%%%%%%%%%%%%%%%%%%%%%%%%%%%%%%%%%%%%%%%%%%%%%%%%%%%%%%%%%%%%%%%%%%%%%%%%%%%%
% Format 
\newcommand{\cR}{\mathcal{R}_S} 		% Random stream
\newcommand{\cQ}{Q_L} 	          	% Localopt Queue

\newcommand{\R}{\mathbb{R}}  % The reals
\newcommand{\cW}{\mathcal{W}} 	% Workers
\newcommand{\cD}{\mathcal{D}} 		% Box-constrained domain
\newcommand{\nw}{n_\mathcal{W}} 	% Number of workers
\newcommand{\tn}{\textnormal}
\newcommand{\maximize}{\operatornamewithlimits{maximize}}
\newcommand{\minimize}{\operatornamewithlimits{minimize}}
\newcommand{\mins}{\operatornamewithlimits{min}}
\newcommand{\argmax}{\operatornamewithlimits{argmax}}
\newcommand{\argmin}{\operatornamewithlimits{argmin}}
\newcommand{\fGamma}{\operatorname{\Gamma}}
\newcommand{\floor}[1] {\lfloor #1 \rfloor}
\newcommand{\ceil}[1] {\lceil #1 \rceil}
\newcommand{\vol}[1] {\operatorname{vol}\left( #1 \right)}
\newcommand{\BigO}[1]{\ensuremath{\operatorname{O}\bigl(#1\bigr)}}
\newcommand{\Matrix}[1]{\begin{bmatrix} #1 \end{bmatrix}}
\newcommand{\Vector}[1]{\Matrix{#1}}

\newcommand*{\SET}[1]  {\ensuremath{\mathcal{#1}}}
\newcommand*{\MAT}[1]  {\ensuremath{\mathbf{#1}}}
\newcommand*{\VEC}[1]  {\ensuremath{\bm{#1}}}
\newcommand*{\CONST}[1]{\ensuremath{\mathit{#1}}}
\newcommand*{\norm}[1]{\mathopen\| #1 \mathclose\|}% use instead of $\|x\|$
\newcommand*{\abs}[1]{\mathopen| #1 \mathclose|}% use instead of $\|x\|$
\newcommand*{\absLR}[1]{\left| #1 \right|}% use instead of $\|x\|$

\def\norm#1{\mathopen\| #1 \mathclose\|}% use instead of $\|x\|$
\newcommand{\normLR}[1]{\left\| #1 \right\|}% use instead of $\|x\|$

% Argonne Logo Colors
\definecolor{ArgonneLogoBlue}{RGB}{4,146,210}
\definecolor{ArgonneLiteBlue}{RGB}{202,214,246}%{176,196,222}
\definecolor{ArgonneLogoRed}{RGB}{228,32,41}
\definecolor{ArgonneLogoGreen}{RGB}{120,202,42}
\definecolor{PMSCoolGray}{RGB}{112,109,110}


\newcommand{\BLUE}[1]{\textcolor{blue}{\bf #1}}
\newcommand{\RED}[1]{\textcolor{ArgonneLogoRed}{\bf #1}}
\newcommand{\PURPLE}[1]{\textcolor{purple}{\bf #1}}
\newcommand{\GREEN}[1]{\textcolor{ArgonneLogoGreen}{\bf #1}}

%%%%%%%%%%%%%%%%%%%%%%%%%%%%%%%%%%%%%%%%%%%%%%%%%%%%%%%%%%%%%%%%%%%%%%%%%%%%%%%%
% Multicol Settings
%%%%%%%%%%%%%%%%%%%%%%%%%%%%%%%%%%%%%%%%%%%%%%%%%%%%%%%%%%%%%%%%%%%%%%%%%%%%%%%%
\setlength{\columnsep}{0.7em}
\setlength{\columnseprule}{0mm}


%%%%%%%%%%%%%%%%%%%%%%%%%%%%%%%%%%%%%%%%%%%%%%%%%%%%%%%%%%%%%%%%%%%%%%%%%%%%%%%%
% Save space in lists. Use this after the opening of the list
%%%%%%%%%%%%%%%%%%%%%%%%%%%%%%%%%%%%%%%%%%%%%%%%%%%%%%%%%%%%%%%%%%%%%%%%%%%%%%%%
\newcommand{\compresslist}{%
\setlength{\itemsep}{1pt}%
\setlength{\parskip}{0pt}%
\setlength{\parsep}{0pt}%
}

\newcommand{\alert}{\RED}
\newcommand{\X}{{\cal X}}
\newcommand{\U}{{\cal U}}
\newcommand{\RA}{$\Rightarrow$}
\newcommand{\ra}{\alert{a}}

\newcommand{\mini}{\mathop{\mbox{minimize}}}
\newcommand{\maxi}{\mathop{\mbox{maximize}}}
\newcommand{\st}{\mbox{subject to}}
\newcommand{\stn}{\mbox{s.t.}}
\newcommand{\dps}{\displaystyle}
\newcommand{\readers}[1]{}

%%%%%%%%%%%%%%%%%%%%%%%%%%%%%%%%%%%%%%%%%%%%%%%%%%%%%%%%%%%%%%%%%%%%%%%%%%%%%%
%%% Begin of Document
%%%%%%%%%%%%%%%%%%%%%%%%%%%%%%%%%%%%%%%%%%%%%%%%%%%%%%%%%%%%%%%%%%%%%%%%%%%%%%

\begin{document}

%%%%%%%%%%%%%%%%%%%%%%%%%%%%%%%%%%%%%%%%%%%%%%%%%%%%%%%%%%%%%%%%%%%%%%%%%%%%%%
%%% Here starts the poster
%%%---------------------------------------------------------------------------
%%% Format it to your taste with the options
%%%%%%%%%%%%%%%%%%%%%%%%%%%%%%%%%%%%%%%%%%%%%%%%%%%%%%%%%%%%%%%%%%%%%%%%%%%%%%
\typeout{Poster Starts}
\background{
  \begin{tikzpicture}[remember picture,overlay]%
    \draw (current page.north west)+(-2em,-0em) node[anchor=north west] {\hspace{-2em}\includegraphics[height=1.1\textheight]{silhouettes_background}};
  \end{tikzpicture}%
}
\definecolor{silver}{cmyk}{0,0,0,0.3}
\definecolor{yellow}{cmyk}{0,0,0.9,0.0}
\definecolor{reddishyellow}{cmyk}{0,0.22,1.0,0.0}
\definecolor{black}{cmyk}{0,0,0.0,1.0}
\definecolor{darkYellow}{cmyk}{0,0,1.0,0.5}
\definecolor{darkSilver}{cmyk}{0,0,0,0.1}

\definecolor{lightyellow}{cmyk}{0,0,0.3,0.0}
\definecolor{lighteryellow}{cmyk}{0,0,0.1,0.0}
\definecolor{lighteryellow}{cmyk}{0,0,0.1,0.0}
\definecolor{lightestyellow}{cmyk}{0,0,0.05,0.0}

\definecolor{KTHBlue}{cmyk}{.71,.37,0.07,0}
\definecolor{KTHsilver}{cmyk}{0,0,0,0.35}
\definecolor{KTHbeige}{cmyk}{0,0.03,0.19,0.04}

% \definecolor{KTHBlue}{RGB}{25,84,166}
\begin{poster}{
  % Show grid to help with alignment
  grid=false,
  columns=3,
  % Column spacing
  colspacing=1em,
  % Color style
  % bgColorOne=ArgonneLogoRed,
  bgColorOne=white,
  bgColorTwo=white,
  borderColor=PMSCoolGray,
  headerColorOne=ArgonneLiteBlue,
  headerColorTwo=ArgonneLogoBlue,
  headerFontColor=black,
  boxColorOne=white,
  boxColorTwo=white,
  % Format of textbox
  textfont=\large,
  textborder=roundedleft,
  % Format of text header
  eyecatcher=true,
  headerborder=open,
  % headerheight=0.08\textheight,
  headerheight=0.14\textheight,
  headershape=roundedright,
  headershade=plain,
%  headershade=shade-lr,
  headerfont=\Large\sf\bf, %Sans Serif
  boxshade=plain,
%  background=shade-tb,
  background=plain,
  linewidth=2pt
  }
  % Eye Catcher
  {\includegraphics[height=6em]{../img/logos/Argonne_cmyk_black.eps}} % No eye catcher for this poster. If an eye catcher is present, the title is centered between eye-catcher and logo.
  % Title
  {\sf $\quad$\\A framework for fully autonomous design of materials via multiobjective optimization and active learning\\$\quad$}
  % Authors
  {\sf  \phantom{\hspace{0em}}
    Tyler Chang$^1$, Jakob Elias$^1$, Stefan Wild$^2$, Santanu Chaudhuri$^{1,3}$ and Joseph Libera$^1$\\

    \smallskip

    {\it \small
    $^1$Argonne National Laboratory $\quad$
    $^2$Lawrence Berkeley National Laboratory $\quad$
    $^3$University of Illinois Chicago\\
    \par
    }

  }
  % Top right logo
  {
    \includegraphics[height=4em]{../img/logos/DOE_logo_color_cmyk.eps}
  }

  \tikzstyle{light shaded}=[top color=baposterBGtwo!30!white,bottom color=baposterBGone!30!white,shading=axis,shading angle=30]

  % Width of left inset image
 \newlength{\leftimgwidth}
 \setlength{\leftimgwidth}{0.78em+8.0em}
 
 \newcounter{boxnum}
 \newcommand{\thebox}{\stepcounter{boxnum} \arabic{boxnum}. }

%%%%%%%%%%%%%%%%%%%%%%%%%%%%%%%%%%%%%%%%%%%%%%%%%%%%%%%%%%%%%%%%%%%%%%%%%%%%%%
%%% Now define the boxes that make up the poster
%%%---------------------------------------------------------------------------
%%% Each box has a name and can be placed absolutely or relatively.
%%% The only inconvenience is that you can only specify a relative position 
%%% towards an already declared box. So if you have a box attached to the 
%%% bottom, one to the top and a third one which should be in between, you 
%%% have to specify the top and bottom boxes before you specify the middle 
%%% box.
%%%%%%%%%%%%%%%%%%%%%%%%%%%%%%%%%%%%%%%%%%%%%%%%%%%%%%%%%%%%%%%%%%%%%%%%%%%%%%
  %
  % A coloured circle useful as a bullet with an adjustably strong filling
  \newcommand{\colouredcircle}[1]{%
    \tikz{\useasboundingbox (-0.2em,-0.32em) rectangle(0.2em,0.32em); \draw[draw=black,fill=baposterBGone!80!black!#1!white,line width=0.03em] (0,0) circle(0.18em);}}

%% %%%%%%%%%%%%%%%%%%%%%%%%%%%%%%%%%%%%%%%%%%%%%%%%%%%%%%%%%%%%%%%%%%%%%%%%%%%%%%
%% \headerbox{}{name=none,column=0,row=0}{
%% %%%%%%%%%%%%%%%%%%%%%%%%%%%%%%%%%%%%%%%%%%%%%%%%%%%%%%%%%%%%%%%%%%%%%%%%%%%%%%
%%   \setlength{\parskip}{0.5\baselineskip}
%%   \begin{itemize}[nosep]
%%     \item text
%%   \end{itemize}
%% }

%%%%%%%%%%%%%%%%%%%%%%%%%%%%%%%%%%%%%%%%%%%%%%%%%%%%%%%%%%%%%%%%%%%%%%%%%%%%%%
\headerbox{Multiobjective Optimization}{name=robust,column=0,row=0}{
%%%%%%%%%%%%%%%%%%%%%%%%%%%%%%%%%%%%%%%%%%%%%%%%%%%%%%%%%%%%%%%%%%%%%%%%%%%%%%
  \setlength{\parskip}{0.5\baselineskip}
  \[
  \begin{array}{lll}
    \displaystyle 
    \mini_{x \in \X}    & F(x) & \alert{F(x) = (f_1(x), f_2(x), \ldots, f_o(x)}\\
    \st      \;  & G(x) \leq 0 & \alert{G(x) = (g_1(x), g_2(x), \ldots, g_p(x)}
  \end{array}
  \]
  \begin{center}
  \includegraphics[width=\textwidth]{../img/moo_old/des-obj-space.png}
  \end{center}
}

%%%%%%%%%%%%%%%%%%%%%%%%%%%%%%%%%%%%%%%%%%%%%%%%%%%%%%%%%%%%%%%%%%%%%%%%%%%%%%
\headerbox{Multiobjective *Simulation* Optimization}{name=AddMan,column=0,below=robust}{
%%%%%%%%%%%%%%%%%%%%%%%%%%%%%%%%%%%%%%%%%%%%%%%%%%%%%%%%%%%%%%%%%%%%%%%%%%%%%%
  \setlength{\parskip}{0.5\baselineskip}
  \[
  \begin{array}{lll}
    \displaystyle 
    \mini_{x \in \X}    & F(x, \BLUE{S(x)}) & \BLUE{S(x) = (s_1(x), s_2(x), \ldots, s_m(x)}\\
    \st      \;  & G(x, \BLUE{S(x)}) \leq 0 & 
    \end{array}
  \]
  \begin{center}
  \includegraphics[width=\textwidth]{../img/moo_old/des-sim-obj-space.png}
  \end{center}
}

%%%%%%%%%%%%%%%%%%%%%%%%%%%%%%%%%%%%%%%%%%%%%%%%%%%%%%%%%%%%%%%%%%%%%%%%%%%%%%
\headerbox{Common Simulation-based Structures}{name=dual,column=0,below=AddMan}{
%%%%%%%%%%%%%%%%%%%%%%%%%%%%%%%%%%%%%%%%%%%%%%%%%%%%%%%%%%%%%%%%%%%%%%%%%%%%%%
  \setlength{\parskip}{0.5\baselineskip}
  
\RED{Sum-of-squared} \BLUE{simulation outputs}:
$$
\min_{x\in {\cal X}} \quad \left(\RED{\sum_{i=1}^{m_1}} \BLUE{S_i(x)}^{\RED{2}},
\quad \RED{\sum_{j=1}^{m_2}} \BLUE{S_j(x)}^{\RED{2}}\right)
$$
\BLUE{One simulation}, \RED{one algebraic} objective:
$$
\min_{x\in {\cal X}} \quad \left(\BLUE{S(x)}, \quad \RED{\sum_{i=1}^n x^2}\right)
$$

}

%%%%%%%%%%%%%%%%%%%%%%%%%%%%%%%%%%%%%%%%%%%%%%%%%%%%%%%%%%%%%%%%%%%%%%%%%%%%%%
\begin{posterbox}[name=ampl,column=1,row=0]{Response Surface Methodology}
%%%%%%%%%%%%%%%%%%%%%%%%%%%%%%%%%%%%%%%%%%%%%%%%%%%%%%%%%%%%%%%%%%%%%%%%%%%%%%
  \setlength{\parskip}{0.5\baselineskip}

  \medskip

  \begin{itemize}
    \item \GREEN{Search/sample} data for raw \BLUE{simulations outputs}
    \item Use \GREEN{surrogates} to model \BLUE{simulations}, \RED{not objectives}
    \item Separately define \RED{objectives and constraints}
    \item Scalarize \RED{objectives} using \GREEN{acquisition functions}
    \item \GREEN{Solve scalarized surrogate problems} and iterate
  \end{itemize}
  \begin{center}
  \includegraphics[width=0.7\textwidth]{../img/moo_old/surrogate-models.pdf}
  \end{center}
\end{posterbox}

%%%%%%%%%%%%%%%%%%%%%%%%%%%%%%%%%%%%%%%%%%%%%%%%%%%%%%%%%%%%%%%%%%%%%%%%%%%%%%
\headerbox{Design Principles}{name=linear,column=1,below=ampl,above=bottom}{
%%%%%%%%%%%%%%%%%%%%%%%%%%%%%%%%%%%%%%%%%%%%%%%%%%%%%%%%%%%%%%%%%%%%%%%%%%%%%%
  \setlength{\parskip}{0.5\baselineskip}
  \begin{tcolorbox}[colback=ArgonneLogoGreen!5!white,colframe=ArgonneLogoGreen!75!black,title=Mix-and-match]
    \begin{itemize}[nosep]
      \item Initial search (design-of-experiments)
      \item Surrogate models
      \item Acquisition/scalarization functions
      \item Scalar optimization solvers
    \end{itemize}
  \end{tcolorbox}
  \begin{tcolorbox}[colback=ArgonneLogoBlue!5!white,colframe=ArgonneLogoBlue!75!black,title=Easy for users and developers]
    \begin{itemize}[nosep]
      \item Support for variety of design vars and simulations
      \item Support various scientific workflows
      \item Embed/extract problems from unit cube
    \end{itemize}
  \end{tcolorbox}
  \begin{tcolorbox}[colback=ArgonneLogoRed!5!white,colframe=ArgonneLogoRed!75!black,title=Flexible problem definitions]
    \begin{itemize}[nosep]
      \item Add design vars, sims, objs, + constraints
      \item Add searches, surrogates, acquisitions, optimizer
      \item Solve serially or in parallel using {\tt libEnsemble} %\cite{libensemble}
    \end{itemize}
  \end{tcolorbox}
}

%%%%%%%%%%%%%%%%%%%%%%%%%%%%%%%%%%%%%%%%%%%%%%%%%%%%%%%%%%%%%%%%%%%%%%%%%%%%%%
\headerbox{A Sample Script}{name=experiments,column=2,row=0}{
%%%%%%%%%%%%%%%%%%%%%%%%%%%%%%%%%%%%%%%%%%%%%%%%%%%%%%%%%%%%%%%%%%%%%%%%%%%%%%
  \setlength{\parskip}{0.1\baselineskip}

  \lstset{style=python}
  \lstinputlisting{quickstart.py}

}

%%%%%%%%%%%%%%%%%%%%%%%%%%%%%%%%%%%%%%%%%%%%%%%%%%%%%%%%%%%%%%%%%%%%%%%%%%%%%%
\headerbox{Download ParMOO}{name=code,column=2,below=experiments}{
%%%%%%%%%%%%%%%%%%%%%%%%%%%%%%%%%%%%%%%%%%%%%%%%%%%%%%%%%%%%%%%%%%%%%%%%%%%%%%
  \setlength{\parskip}{0.5\baselineskip}
  \begin{itemize}[nosep]
    \item {\tt git clone https://github.com/parmoo/parmoo}
    \item {\tt pip install parmoo}
  \end{itemize}
  \begin{center}
  \includegraphics[width=0.38\textwidth]{../img/logos/logo-ParMOO.png}
  \hskip 24pt
  \includegraphics[width=0.18\textwidth]{../img/logos/logo-py.png}
  \hskip 24pt
  \includegraphics[width=0.08\textwidth]{../img/logos/logo-gh.png}
  \end{center}
}

%%%%%%%%%%%%%%%%%%%%%%%%%%%%%%%%%%%%%%%%%%%%%%%%%%%%%%%%%%%%%%%%%%%%%%%%%%%%%%
\headerbox{Continuing Work}{name=future,column=2,below=code, above=bottom}{
%%%%%%%%%%%%%%%%%%%%%%%%%%%%%%%%%%%%%%%%%%%%%%%%%%%%%%%%%%%%%%%%%%%%%%%%%%%%%%
  \setlength{\parskip}{0.5\baselineskip}
  \begin{itemize}[nosep]
    \item Continue to add new solvers and techniques
    \item Support wider variety of problems \& workflows
  \end{itemize}

}

%%%%%%%%%%%%%%%%%%%%%%%%%%%%%%%%%%%%%%%%%%%%%%%%%%%%%%%%%%%%%%%%%%%%%%%%%%%%%%%
%\headerbox{References}{name=test,column=2,below=future,above=bottom}{
%%%%%%%%%%%%%%%%%%%%%%%%%%%%%%%%%%%%%%%%%%%%%%%%%%%%%%%%%%%%%%%%%%%%%%%%%%%%%%%
%  \setlength{\parskip}{0.5\baselineskip}
%  \scriptsize
%  \bibliographystyle{abbrv}
%  \begingroup
%  \renewcommand{\section}[2]{}
%  \bibliography{paper}
%  \endgroup
%}

\end{poster}%
%
\end{document}
