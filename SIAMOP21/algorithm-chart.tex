\documentclass{article}
\usepackage{geometry}
\geometry{
paperwidth=8.5in,
paperheight=5.7in,
textwidth=7.5in,
left=0.5in,
textheight=4.7in,
top=0.5in
}

\usepackage[latin1]{inputenc}
\usepackage{tikz}
\usetikzlibrary{shapes,arrows}
\begin{document}
\pagestyle{empty}


% Define block styles
\tikzstyle{decision} = [diamond, draw, fill=blue!20, 
    text width=4.5em, text badly centered, node distance=3cm, inner sep=0pt]
\tikzstyle{block} = [rectangle, draw, fill=blue!20, 
    text width=10em, text centered, rounded corners, minimum height=4em]
\tikzstyle{line} = [draw, -latex']
\tikzstyle{cloud} = [draw, ellipse,fill=red!20, node distance=3cm,
    minimum height=2em]
    
\begin{tikzpicture}[node distance = 2.5cm, auto]
    % Place nodes
    \node [block] (search0)
	{initial {\tt GlobalSearch} to generate simulation data in ${\cal X}$};
    \node [block, below of=search0] (fit0)
	{fit a {\tt SurrogateFunction} to each of $m$ simulation outputs};
    \node [block, right of=search0, node distance=20em] (id)
	{use $q$ {\tt AcquisitionFunction} to set $q$ new targets};
    \node [block, below of=id] (optk)
    {use {\tt SurrogateOptimizer} to generate $q$ candidate solutions
    (one for each scalarization)};
    \node [block, below of=optk] (evalk)
	{evaluate simulations at each candidate solution};
    \node [block, below of=evalk] (fitk)
    {update the {\tt SurrogateFunction} models};
    \node [decision, right of=optk, node distance=20em] (decide)
	{sim budget done?};
    \node [block, below of=decide, node distance=3cm] (exit)
	{construct the final nondominated and efficient set and return};
    % Draw edges
    \path [line] (search0) -- (fit0);
    \path [line] (fit0) -- (id);
    \path [line] (id) -- (optk);
    \path [line] (optk) -- (evalk);
    \path [line] (evalk) -- (fitk);
    \path [line] (fitk) -- (decide);
    \path [line] (decide) -- node [near start] {no} (id);
    \path [line] (decide) -- node {yes} (exit);
\end{tikzpicture}


\end{document}

