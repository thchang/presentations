% preamble
\documentclass[11pt]{article}
\usepackage{geometry, setspace, multicol}
\geometry{
letterpaper,
total={6.5in,9in},
left=1in,
top=1in}
\parindent = 0pt
\parskip = 6pt

% document
\begin{document}

% headers
\begin{centering}
\textbf{Surrogate Modeling of Simulations for Multiobjective Optimization Applications}
\end{centering}
\begin{flushright}
Tyler H.\ Chang\\
\today\\
Abstract for SIAM OP21
\end{flushright}

% body
Many real-world blackbox multiobjective optimization problems are derived
from complex numerical simulations.
However, the objectives may not be the direct result of simulations,
but rather algebraic functions of simulation outputs.
As an example, consider the case where numerous simulation outputs are
being minimized, and they are condensed into a smaller number of objectives
by applying several algebraic utility functions (e.g., sum-of-squares).
One could consider each objective as an individual blackbox function.
However, this approach would require each contributing simulation to be
evaluated each time that a utility function is evaluated.
Furthermore, this approach would lose precious information about the
structure of the utility functions.
In this talk, we consider the advantages (both computational and theoretical)
of considering simulations separately from objectives, specifically, in
the context of surrogate modeling.

\end{document}
