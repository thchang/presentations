% Template for overheads, landscape, fancy VT footlines.
%
\magnification=1800 \rightskip=0pt plus 5em
\hsize=9.0truein \vsize=6.5truein
\parindent=0pt \parskip=0pt plus 4truept
\baselineskip=21truept plus 3truept minus 2truept \lineskiplimit=3truept
\lineskip=3truept plus 3truept \tolerance=1000

%fonts
\font\helvr=phvr at 18truept
\font\helvb=phvb at 18truept
\font\timesi=ptmri at 18truept
\font\helvbXX=phvb at 20truept
\font\helvbXXIV=phvb at 24truept
\font\newfont=pbkli

\def\rm{\fam=0\helvr} \let\bf=\helvb \let\tt=\helvb \let\it=\timesi
\let\sl=\timesi

%definitions
\def\cl#1{\centerline{#1}}
\def\head#1{\centerline{\helvbXXIV#1}}
{\obeyspaces\global\let =\ }
\def\verbatim{\parindent=0pt\tt\obeylines\obeyspaces}
\def\l{\hfill\&}        
\def\shift{\hsize=8.5truein\parindent=25truept\indent}
\def\bull{\par\hangindent=20pt\hangafter=1
\indent\hbox to 20pt{\hfil$\bullet$ }\ignorespaces}

% for spacing in tables
\newdimen\digitwidth \setbox0=\hbox{\rm0} \digitwidth=\wd0

\input epsf % For Postscript figures:{\epsfxsize=  \epsffile{}}
\helvr

%**********************************  Page 1 ******************************

\cl{\helvbXXIV A FORTRAN 90 GENETIC ALGORITHM MODULE}\smallskip
\cl{\helvbXXIV FOR DESIGN OF COMPOSITE LAMINATE STRUCTURES}
\bigskip
\bigskip
\cl{M. T. McMahon and L. T. Watson}
\cl{Departments of Computer Science and Mathematics}
\cl{Virginia Polytechnic Institute and State University}
\cl{Blacksburg, VA 24061-0106  USA} 
\medskip
\cl{G. A. Soremekun and Z. G\"urdal}
\cl{Department of Engineering Science and Mechanics}
\cl{Virginia Polytechnic Institute and State University}
\cl{Blacksburg, VA 24061-0219 USA}
\medskip
\cl{R. T. Haftka}
\cl{Department of Aerospace Engineering, Mechanics, and Engineering Science} 
\cl{University of Florida}
\cl{Gainesville, FL 32611-6250  USA}
\vfil
\cl{\bf http://csgrad.cs.vt.edu/\char'176mcmahon/gamod.ps}
\bigskip
\bigskip
\vtop to 0pt{\vskip -30truept
\cl{\epsfxsize=1in \epsffile{VPIlogo.ps}}\vss}
\footline={\hfil}
\eject
%**********************************  Page 2 ******************************
\footline={\fiverm\folio\quad\leaders\hrule
height3truept\hfil\quad\newfont Virginia Tech}

\head{The goal.}
\bigskip \bigskip
To develop a genetic algorithm tool for composite laminate structural
design optimization that can:
\medskip
\bull be used for a wide variety of structures,
\bull be integrated with analysis codes easily,
\bull facilitate development and testing of new genetic algorithms.
\vfil
\head{The target problem.}
\bigskip\bigskip
A stiffened panel made from an assemblage of laminated composite plates, where
each plate is composed of several plies made of one or more fiber
reinforced plastic materials.
\medskip
{\bf Discrete variables}: material and orientation of the fibers.
\medskip
{\bf Continuous design variables}: geometric dimensions of the plates.
\vfil\eject

\head{The test problem.}
\bigskip\bigskip
\bull 36 in. long by 30 in. wide, simply supported panel.
\bull Loaded under any combination of axial and shear loads.
\bull Two material types: graphite-epoxy and Kevlar-epoxy.
\bull Ply orientations from  -75 to 90 degrees, in increments
of 15 degrees.
\bull A fitness function which is a measure of the weight, penalized
by small or violated safety margins.

\vfil
%\centerline{\epsfxsize=5.0truein\epsffile{laminate.eps}}
\vskip-12pt
\eject

%**********************************  Page 3 ******************************

\head{The structure of a population.}
\vfil
%\centerline{\epsfxsize=7.0truein\epsffile{population2.eps}}
\vfil
\eject

%**********************************  Page 4 ******************************
\head{Why Fortran 90?  Abstract data types.}
\bigskip
Data representations reflect underlying concepts better.
\medskip
Inheritance allows a data type definition to use other data types.
\medskip
e.g., the data types for an individual are built with:
\bigskip
{\verbatim
! Gene data types:
    type ply\_gene
        integer (KIND=small)  ::orientation
        integer (KIND=small)  ::material
    end type ply\_gene
\medskip
    type geometry\_gene
        real (KIND=R8)  ::digit
    end type geometry\_gene
}\vfil\eject
\head{Why Fortran 90?  Abstract data types.}
\bigskip
{\verbatim
! Chromosome data types:
type laminate\_chromosome
    type (ply\_gene), pointer, dimension(:):: ply\_array
end type laminate\_chromosome
!
type geometry\_chromosome
    type (geometry\_gene), pointer, dimension(:):: geometry\_gene\_array
end type geometry\_chromosome
!
! Individual (structure) data type:
type individual
    type (laminate\_chromosome),pointer,dimension(:):: laminate\_array
    type (geometry\_chromosome),pointer,dimension(:):: geometry\_array
end type individual
}
\vfil\eject

%**********************************  Page 5 ******************************
\head{Why Fortran 90?  Dynamic Memory Allocation.}
\bigskip
Array sizes can be specified at program run time, which
supports structure designs and populations of variable size.
\medskip
e.g., a child individual's laminate array is sized with:
\bigskip
{\shift\vbox
{\verbatim
!Allocate laminates:
type (individual) :: child
allocate (child\%laminate\_array(individual\_size\_lam))
}
}
\medskip
and deallocated with:
\medskip
{\shift\vbox
{\verbatim
!Deallocate laminates:
deallocate (child\%laminate\_array)
}
}
\bigskip
Array dimensions, and thus memory usage, depend on user input.
 
\vfil \eject
%**********************************  Page 6 ******************************
\head{Why Fortran 90?  High Level Array Operations.}
\bigskip
Operations which were previously confined to scalar data elements can be 
performed on arrays:
\medskip
{\shift\vbox
{\verbatim
! Declare ply array:
integer, dimension (3,10) :: ply\_array
$\vdots$
! Initialize ply\_array to zero:
ply\_array = 0
}
}
\medskip
Portions of an array may be accessed as a single object.
Array programming becomes more efficient and code more readable.
e.g., single-point crossover can be implemented with:
\medskip
{\shift\vbox
{\verbatim
type (laminate\_chromosome) :: child, parent(2)
$\vdots$
!Crossover from parent 1:
child\%ply\_array(1:cross\_point) = parent(1)\%ply\_array(1:cross\_point)
!Crossover from parent 2:
child\%ply\_array(cross\_point+1:) = parent(2)\%ply\_array(cross\_point+1:)
}
}
\vfil\eject

%**********************************  Page 7 ******************************
\head{Why Fortran 90?  Modularity.}
\bigskip
Encapsulation---data type declarations and related subprograms are kept
in a separate syntactic unit.
\medskip
Code maintainability and reusability are enhanced---the GA implementation is
separate from the specific optimization problem.
\medskip
e.g., the GENERIC\_GA module is introduced into a user program with:
\bigskip
{\shift\vbox
{\verbatim
PROGRAM optimize
USE GENERIC\_GA
$\vdots$
END PROGRAM optimize
}
}
\medskip
or specific portions of the module are introduced with:
\medskip
{\shift\vbox
{\verbatim
PROGRAM optimize
USE GENERIC\_GA, ONLY : population, initialize\_population
$\vdots$
END PROGRAM optimize
}
}

\vfil\eject
%**********************************  Page 8 ******************************

\head{Fortran 90 program units.}
\vfil
%\centerline{\epsfxsize=6.0truein\epsffile{progunits.eps}}
\vfil\eject

%**********************************  Page 9 ******************************

\head{The GA package of operators.}
\bigskip
The GA package written for use with the GENERIC\_GA module contains the
following genetic operators:
\bull crossover---one point and two point,
\bull mutation,
\bull ply addition,
\bull ply deletion,
\bull  ply permutation.
\medskip
By using the GA module, the package operators are guaranteed to work
with any composite laminate structure.

\vfil\hrule
\bigskip
Test runs used an elitist selection scheme and operator probabilities of
crossover $=1.00$, ply mutation $=0.05$, orientation mutation $=0.05$,
addition $=0.05$, deletion $=0.10$, and permutation $=0.25$.
\eject

%**********************************  Page 10 ******************************

\head{The GA package continuous variable crossover.}
\bigskip\bigskip
This operator allows optimization with continuous variables not represented
by discrete genes.
Given parents $p_1$ and $p_2$, the geometry gene crossover first creates a
randomly distributed $N(\mu,\sigma)$ real number $r$, where
$$\mu={p_1+p_2\over2},\qquad \sigma={|p_1-p_2|\over2},  $$
and then enforces physical limits by taking the child value as
$$ c=\min\{\max \{r,L\},U\},$$
where $L$, $U$ are lower, upper limits on the geometry variable.  Note
that with probability 0.68 (the probability that a normal variate lies
within one standard deviation of the mean $\mu$), $c$ lies between the
parent values $p_1$ and $p_2$, but that $c$ can also be well outside
the segment $[p_1,p_2]$.  This is essentially how evolutionary
algorithms (B\"ack, 1996) work, where the value of $\sigma$ itself can
be adapted as the evolution proceeds.

\vfil\eject

%**********************************  Page 11 ******************************
\head{Conclusions.}
\bigskip
The design effort described has yielded:
\bull a flexible, general, reusable GA optimization tool,
\bull Fortran 90 compliance along with compatibility
with existing FORTRAN 77 analysis codes,
\bull optimal designs which match those from custom designed,
problem specific GA implementations.
\vfil\hrule\bigskip
\head{Future work.}
\bigskip
\bull expansion of GA package---additional selection schemes,
subpopulation interaction (migration).
\bull distributed memory parallel implementation.

\vfil\eject\bye
