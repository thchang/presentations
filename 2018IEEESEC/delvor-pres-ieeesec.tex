% Template for overheads, landscape, fancy VT footlines.
%
\magnification=1800 \rightskip=0pt plus 5em
\hsize=9.0truein \vsize=6.5truein
\parindent=0pt \parskip=0pt plus 4truept
\baselineskip=21truept plus 3truept minus 2truept \lineskiplimit=3truept
\lineskip=3truept plus 3truept \tolerance=1000

%fonts
\font\helvr=phvr at 18truept
\font\helvb=phvb at 18truept
\font\timesi=ptmri at 18truept
\font\helvbXX=phvb at 20truept
\font\helvbXXIV=phvb at 24truept
\font\newfont=pbkli

% bb font : {\bb [text]}
\font\tenamsb=msbm10
\font\sevenamsb=msbm7
\font\fiveamsb=msbm5
\newfam\bbfam
\textfont\bbfam=\tenamsb
\scriptfont\bbfam=\sevenamsb
\scriptscriptfont\bbfam=\fiveamsb
\def\bb{\fam\bbfam}

\def\rm{\fam=0\helvr} \let\bf=\helvb \let\tt=\helvb \let\it=\timesi
\let\sl=\timesi

%definitions
\def\cl#1{\centerline{#1}}
\def\head#1{\centerline{\helvbXXIV#1}}
{\obeyspaces\global\let =\ }
\def\verbatim{\parindent=0pt\tt\obeylines\obeyspaces}
\def\l{\hfill\&}        
\def\shift{\hsize=8.5truein\parindent=25truept\indent}
\def\bull{\par\hangindent=20pt\hangafter=1
\indent\hbox to 20pt{\hfil$\bullet$ }\ignorespaces}

% for spacing in tables
\newdimen\digitwidth \setbox0=\hbox{\rm0} \digitwidth=\wd0

\input epsf % For Postscript figures:{\epsfxsize=  \epsffile{}}
\helvr

%**********************************  Page 1 ******************************

\cl{\helvbXXIV COMPUTING THE UMBRELLA NEIGHBOURHOOD OF}\smallskip
\cl{\helvbXXIV A VERTEX IN THE DELAUNAY TRIANGULATION AND}\smallskip
\cl{\helvbXXIV A SINGLE VORONOI CELL IN ARBITRARY DIMENSION}
\bigskip
\bigskip
\cl{\bf Tyler Chang$^a$, Layne Watson$^{abc}$, Thomas Lux$^a$,}
\cl{\bf Sharath Raghvendra$^a$, Bo Li$^a$, Li Xu$^d$, Ali Butt$^a$,}
\cl{\bf Kirk Cameron$^a$, and Yili Hong$^d$}
\bigskip
\bigskip
\cl{Departments of Computer Science$^a$, Mathematics$^b$,}
\cl{Aerospace \& Ocean Engineering$^c$, and Statistics$^d$}
\bigskip
\cl{Virginia Polytechnic Institute and State University}
\cl{Blacksburg, VA 24061-0106  USA} 
\vfil
\vtop to 0pt{\vskip -30truept
\cl{\epsfxsize=1in \epsffile{VT_Logo_CompSci.eps}}\vss}
\footline={\hfil}
\eject

%**********************************  Page 2 ******************************
\footline={\fiverm\folio\quad\leaders\hrule
height3truept\hfil\quad{\epsfxsize=1in \epsffile{VT_Logo_CompSci.eps}}}%\newfont Virginia Tech}

\head{What is a Tessellation?}
\bigskip\bigskip
A $d$-dimensional {\it tessellation} of some convex region 
$X \subseteq {\bb R}^d$ is a division of $X$ into closed (typically convex)
sets called cells such that:
\smallskip
\bull They are disjoint except along their shared boundaries.
\smallskip
\bull Their union is $X$.
\bigskip
\cl{\hbox{\epsfxsize=25truepc\epsffile{tessplane.eps}}}
\vfil\eject


%**********************************  Page 3 ******************************
\head{What is a Triangulation?}
\bigskip\bigskip
A $d$-dimensional {\it triangulation} $T$ of a (finite) set of points 
$P$ in ${\bb R}^d$ is a tessellation of the {\it convex hull} of $P$, such 
that:
\smallskip
\bull Each cell in $T$ is a $d$-simplex.
\smallskip
\bull The vertices of each simplex are points in $P$.
\bigskip
\cl{\hbox{\epsfxsize=25truepc\epsffile{triangleplane.eps}}}
\vfil\eject

%**********************************  Page 4 ******************************
\head{Who Cares?}
\bigskip \bigskip
\bull Discretizing space.
\medskip
\bull Interpolation and mesh generation.
\medskip
\bull Computer vision/graphics.
\medskip
\bull Topological data anlaysis ($\alpha$-shapes and $k$-skeletons).
\medskip
\bull Notions of ``nearness'' and ``neighbors'' in high-dimensions.
\medskip
\cl{\hbox{\epsfxsize=20truepc\epsffile{seamount.eps}}}
\vfil
\eject

%**********************************  Page 5 ******************************
\head{The Voronoi Diagram and Delaunay Triangulation}
\bigskip \bigskip
Given points $P \in {\bb R}^d$, the Voronoi diagram is a ``nearest neighbor''
tessellation.
\medskip
The Delaunay triangulation is its {\it geometric dual}.
\bigskip
\cl{\hbox{\epsfxsize=30truepc\epsffile{del-vor.eps}}}
\vfil\eject

%**********************************  Page 6 ******************************
\head{The Empty Circumsphere Property}
\bigskip \bigskip
More usefully, the Delaunay triangulation can be defined in terms of
its {\it empty circumsphere property}:
\medskip
\cl{\hbox{\epsfxsize=30truepc\epsffile{delaunaycircle.eps}}}
\medskip
Note that the center of each circumsphere is a Voronoi vertex.
\vfil \eject

%**********************************  Page 7 ******************************
\head{Existence and Uniqueness}
\bigskip \bigskip
\bull Trivially, the Voronoi diagram always exists and is unique.
\medskip
\bull Since the Delaunay triangulation is defined in terms of the Voronoi 
diagram, it exists so long as not all the points in $P$ lie in an 
{\it affine subspace}.
\medskip
\bull Unique if no $d+2$ or more points lie on the same circumsphere.
\medskip
\bull Since the probability of degeneracy is zero, assume the points are in 
{\it general position} and the Delaunay triangulation exists and is unique.
\vfil \eject

%**********************************  Page 8 ******************************
\head{The Curse of Dimensionality}
\bigskip \bigskip
Both can be reasonably computed in two or three dimensions
\bigskip
Unfortunately both the Delaunay triangulation and the Voronoi
diagram grow exponentially in {\it size} with respect to the dimension.
\bigskip
In the worst case, the number of Voronoi vertices/Delaunay simplices for
a set of $n$ points in ${\bb R}^d$ is
$${\cal O}(n^{\lceil d/2 \rceil}).$$
\bigskip
That means that the computational time {\it and} space complexity {\it must}
grow {\it exponentially}! 
\vfil\eject

%**********************************  Page 9 ******************************
\head{Umbrella Neighborhoods and Single Voronoi Cells}
\bigskip \bigskip
The Delaunay {\it umbrella neighborhood} of a point $p \in P$ is the
set of simplices ``incident'' at $P$.
\smallskip
I.e., the set of simplices with $P$ as a vertex.
\bigskip
Since the circumcenter of each simplex is a Voronoi vertex and vice versa,
computing the umbrella neighborhood of $p$ is equivalent to computing the
{\it Voronoi cell} of $p$.
\medskip
\cl{\hbox{\epsfxsize=30truepc\epsffile{umbrella.eps}}}
\vfil\eject

%**********************************  Page 10 ******************************
\head{Computing the Umbrella Neighborhood}
\bigskip\bigskip
Applications that only require the umbrella neighborhood/a single Voronoi
cell:
\smallskip
\bull Multiobjective optimization: Well-spacedness in parameter space
(Deshpande, Watson, and Canfield, 2016).
\smallskip
\bull Modelling of porosity in a pebble-bed nuclear reactor: How much space
between each pebble?
(Rycroft et al., 2006).
\vfil
\head{Research Questions}
\bigskip\bigskip
Can one compute just an umbrella neighborhood/single cell without computing
the whole triangulation/diagram?
\bigskip
How much time will that save?
\vfil\eject

%**********************************  Page 11 ******************************
\head{New Algorithm for Computing the Umbrella Neighbourhood}
\bigskip \bigskip
\bull Grow an initial simplex from $p$ (through a sequence of least squares
problems).
\medskip
\bull Flip accross each facet containing $p$ as a vertex to ``close'' them.
\medskip
\bull Keep track of all ``open'' facets containing $p$ as a vertex in a data 
structure (called the ``AFL'').
\medskip
\bull Iterate until all open facets have been closed.
\smallskip
\cl{\hbox{\epsfxsize=20truepc\epsffile{DelaunayWalk.eps}}}
\vfil\eject

%**********************************  Page 12 ******************************
\head{Algorithm Analysis}
\bigskip \bigskip
\bull Takes ${\cal O}\big(n d^4\big)$ time to compute the first simplex.
\medskip
\bull Takes ${\cal O}\big(n d^3\big)$ time to perform each ``flip.''
\medskip
\bull If there are $k$ simplices in the umbrella neighbourhood, must
perform $k$ ``flips.''
\medskip
\bull If $k << |DT(P)|$, this will save a significant amount of compute 
time/space!
\vfil\eject

%**********************************  Page 13 ******************************
\head{Results: Empirical Analysis}
\bigskip \bigskip
\cl{Average run times in seconds for sizes $n$ and dimensions $d$}
\cl{(with a sample size of 20, for randomly generated data)}
\medskip
\offinterlineskip \tabskip=0pt
\def\filler{height2pt&\omit&&\omit&&\omit&&\omit&&\omit&\cr}
\def\tablerule{\noalign{\hrule}}
\centerline{\vbox{
\halign{&\vrule#&\strut\hskip 4pt\hfil#\hskip 4pt\cr
\tablerule\filler
&\hfill&&$n=2K$\hfill&&$n=8K$\hfill&&$n=16K$\hfill&&$n=32K$\hfill&\cr
\filler\tablerule\filler
&\omit\hskip 4pt\hbox{$d=2$}
&&0.1 s\hfill&&1.6 s\hfill&&6.3 s\hfill&&25.0 s\hfill&\cr
&\omit\hskip 4pt\hbox{$d=3$}
&&0.1 s\hfill&&1.8 s\hfill&&7.0 s\hfill&&27.8 s\hfill&\cr
&\omit\hskip 4pt\hbox{$d=4$}
&&0.2 s\hfill&&2.0 s\hfill&&7.6 s\hfill&&30.0 s\hfill&\cr
&\omit\hskip 4pt\hbox{$d=5$}
&&0.3 s\hfill&&2.6 s\hfill&&9.2 s\hfill&&34.1 s\hfill&\cr
\filler\tablerule}}}
\bigskip\medskip
\cl{Average number of simplices in the umbrella neighbourhood}
\medskip
\offinterlineskip \tabskip=0pt
\def\filler{height2pt&\omit&&\omit&&\omit&&\omit&&\omit&\cr}
\def\tablerule{\noalign{\hrule}}
\centerline{\vbox{
\halign{&\vrule#&\strut\hskip 4pt\hfil#\hskip 4pt\cr
\tablerule\filler
&\hfill&&$n=2K$\hfill&&$n=8K$\hfill&&$n=16K$\hfill&&$n=32K$\hfill&\cr
\filler\tablerule\filler
&\omit\hskip 4pt\hbox{$d=2$}
&&6.30\hfill&&5.90\hfill&&6.25\hfill&&6.65\hfill&\cr
&\omit\hskip 4pt\hbox{$d=3$}
&&28.10\hfill&&28.90\hfill&&29.50\hfill&&29.10\hfill&\cr
&\omit\hskip 4pt\hbox{$d=4$}
&&151.90\hfill&&170.55\hfill&&173.60\hfill&&161.00\hfill&\cr
&\omit\hskip 4pt\hbox{$d=5$}
&&1122.90\hfill&&1115.70\hfill&&1111.00\hfill&&1038.70\hfill&\cr
\filler\tablerule}}}
\vfil\eject

%**********************************  Page 14 ******************************
\head{Issues in Stability}
\bigskip \bigskip
Thus far have only considered points in {\it general position}.
In the real world, degeneracies happen, and this can be disastrous for
the algorithm.
\smallskip
\bull The only known solution is perturbation.
\bigskip
\cl{\hbox{\epsfxsize=30truepc\epsffile{degen.eps}}}
\vfil\eject

%**********************************  Page 15 ******************************
\head{Conclusion}
\bigskip \bigskip
A new scalable algorithm for computing Delaunay umbrella neighbourhoods /
Voronoi diagrams was introduced.
\medskip
Empirically, it is able to save a significant amount of time by computing
a small subset of the complete triangulation.
\medskip
Unfortunately, it has trouble handling degenerate data.
\vfil
\head{Future Work}
\bigskip \bigskip
Investigate perturbation strategies and their stability guarantees.
\medskip
Compare to {\verbatim Voro++} library for computing a single Voronoi cell
in $3$-dimensions.
\vfil\eject

\bye
